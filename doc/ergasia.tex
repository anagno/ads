%xelatex -shell-escape -output-directory=bin ergasia.tex
\documentclass{assignment}

\usepackage{enumerate} % Για την χρησιμοποίηση roman enumerate
\usepackage{pdflscape}

%\university{Πανεπιστήμιο Πειραιώς}{Πα.Πει.}
%\school{Τμήμα Πληροφορικής}{Π.Μ.Σ. "Πληροφορική"}
%\department{Πρόγραμμα Μεταπτυχιακών Σπουδών «Πληροφορική»}{}
%\cover{images/cover.jpg}{http://www.cyberciti.biz/faq/grub-boot-into-single-user-mode/}

\title{Αντικειμενοστρεφής Προγραμματισμός  \\ Εργασία Εξαμήνου}
%\projectlevel{Εργαστήριο Λειτουργικά Συστήματα}
%\lesson{Λειτουργικά Συστήματα}{1}
\date{Αθήνα, 2014}

\author{Αναγνωστόπουλος Βασίλης - Θάνος}
%\register{ΜΠΠΛ13002}{1}

%\exercauthor{Αναγνωστόπουλος Βασίλης - Θάνος}{06107083}{9}

%\advisor{Τσακίρη Μαρία, Αναπληρώτρια Καθηγήτρια Ε.Μ.Π.}

\begin{document}

\maketitle
% Να σκεφτώ τί αλλαγές θέλω να κάνω με τις αριθμήσεις και άμα θέλω να κάνω.
% Να σκεφτώ να τις ενσωματώσω και στο assignment.cls

%\setcounter{page}{1} 
%\pagenumbering{roman}

%\pagestyle{plain}
%\tableofcontents
%\listoftables
%\listoffigures
%\newpage


%\pagestyle{headings}
%\pagestyle{fancy}
%\setcounter{page}{1} 
%\pagenumbering{arabic}

\section{Εισαγωγή - Εκφώνηση της άσκησης}

Οι αγγελίες σε εφημερίδα χρεώνονται κλιμακωτά ανάλογα με το πλήθος των γραμμάτων σύμφωνα με τον παρακάτω πίνακα:

\begin{table}[h]
\begin{center}
  \begin{tabular}{|c|c|}
    \hline
    Πλήθος Γραμμάτων & Κόστος ανα λέξη \\ \hline
    1-25             & 5               \\ \hline 
    26-100           & 3.5             \\ \hline 
    101 και άνω      & 2               \\ \hline  
  \end{tabular}
\caption{Τα εικονίδια του διαγράμματος οντοτήτων συσχετίσεων.}
\label{table:icons}
\end{center}
\end{table}

Να γράψετε πρόγραμμα  JAVA:
\begin{itemize}
  \item Θα διαβάσει το κείμενο μιας αγγελίας γράμμα-γράμμα έως ότου να συναντήσει την τελεία.
  \item Θα υπολογίζει πόσα γράμματα και πόσες λέξεις περιέχει η αγγελία;
  \item Θα εμφανίζει το ποσό της χρέωσης για την αγγελία.
  \item Αν στην ειδική στήλη της εφημερίδας που τοποθετούνται οι αγγελίες χωρούν σε κάθε γραμμή 11 γράμματα κατά μέσο όρο, να εμφανίζει πόσες γραμμές είναι η αγγελία.
\end{itemize}

Σημείωση: Να θεωρήσετε ότι σε κάθε επανάληψη εισάγεται ένας μόνο χαρακτήρας και δεν χρησιμοποιούνται άλλα σημεία στίξης πέραν της τελείας. Οι λέξεις χωρίζονται με το χαρακτήρα του κενού.

\section{Υλοποίηση του προγράμματος}

\inputminted[breaklines=true,linenos,tabsize=2]{java}{../../src/ads/Ads.java}




\phantomsection \label{Βιβλιογραφία}
\addcontentsline{toc}{section}{Βιβλιογραφία}
%\mtcaddchapter[Βιβλιογραφία] % Λόγω του minitoc
\bibliographystyle{plain}
\bibliography{references}

\newpage

\end{document}

